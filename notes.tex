\documentclass[11pt,a4paper,twocolumn]{article}

\usepackage{lipsum}
\usepackage{graphicx}
\usepackage{physics}
\usepackage{amsmath}

\begin{document}
\title{Elementar-Teilchen-Theorie Notizen}
\author{Maximilian Sackel}
\date{\today}
\maketitle

\section{Einleitung}
Elementarteilchenphysik bedeutet so viel wie die Suche nach neuen fudmanentalen
Naturgesetzen. 
Zur Beschreibung der Physik werden relativistische Quantenfelder genutzt.
Das wichtigste Modell in der Teilchenphysik ist das 'Standardmodell'.
\begin{figure}[h]
		\centering
		\includegraphics[width=0.4\textwidth]{./picture/Quark.png}
\end{figure}
Teilchenbahnen koennen aufgrund der QM nicht beobachtet werden. 
Die Anzahl an moeglichen Prozessen ist durch die Symmetrie eingeschraenkt.
(Raeummliche Trans./Impuls, zeitl. Trans./Energie, Drehung/Drehimpus, zufaell.
Sym...)

\subsection{Lorentztransformation}
Naturgesetze sind invariant unter Lorentztrafo.
\begin{equation}
		x \cdot y = \sum_{\mu \nu} x^{\mu} y^{nu} g_{\mu \nu}
\end{equation}
Dies laesst sich mittels eines Lorentzboost $x^{\mu} \to x'^{\mu} =
\Lambda^{\mu}_{\nu} x^{\nu} $ pruefen.
\begin{equation}
		x' \cdot y' = \Lambda^{\mu}_{\nu} x^{\nu} \cdot \Lambda^{\rho}_{\sigma}
		x^{\sigma} \cdot g_{\mu \nu} = x^{\nu} y^{\sigma} g_{\nu \sigma}
\end{equation}
Dazu muss der entsprechende metrische Tensor entsprechend $g_{\mu \nu}
\Lambda^{\mu}_{\rho} \Lambda^{\nu}_{\delta} \to g_{\rho \delta}$ tansformieren. 

\subsection{Kinematik von Streuprozessen}
Das Impulsquadrat ist allgemein in der ART eine invariante.
Fuer einen Prozess $p_a + p_b \to p_c + p_d$ laesst sich die Energieerhaltung
\begin{equation}
		p_a^0 + p_b^0 = p_c^0 + p_d^0 
\end{equation}
sowie Impulserhaltung 
\begin{equation}
		\vec{p_a} + \vec{p_b} = 0 = \vec{p_c} + \vec{p_d}
\end{equation}
im Schwerpunktsystem (CMS) definieren.
Dazu lassen sich die Lorentzinvarianten kinematischen Groessen/ 
Mandelstammvariablen definieren.
\begin{eqnarray}
		s = (p_a + p_b)^2 \\
		t = (p_a - p_c)^2 \\
		u = (p_a - p_d)^2
\end{eqnarray}
Diese sind linear abhaengig,
\begin{equation}
		(u + t + s) = \sum_i m_i
\end{equation}
sodass es nur zwei linear unabheanigige Variablen gibt.

\subsection{Integration auf der Massenschale}
Das Lorentzinvariante Integral einer beliebigen sklaraen Funktion $f(x)$ 
\begin{equation}
		\int d^4p \ \delta(p^2 -M^2) \theta(p^0) f(p)
\end{equation}
kann ueber die vierer Impulse $p^2 > 0, p^0 > 0$ kann geschrieben werden als
\begin{equation}
		\int d^3p \ \frac{f(\sqrt{\vec{p}^2 + M^2}, \vec{p})}{2\sqrt{\vec{p}^2
		+ M^2}}
\end{equation}
Zum beweis der Entwicklung wird die entwicklugn der Delta-Funktion genutzt wobei
$x_0$ die Nullstellen der Funktion $f$ sind.
\begin{equation}
		\delta(f(x)) = \frac{\delta(x-x_0)}{|f'(x_0)|}
\end{equation}

\section{Streutheorie}
\subsection{S-Matrix}
Man nehme an, der Hamiltonoperator bestehe aus einem Term $H_0$ welche ein 
beliebige Anzahl freier Teilchen beschreibt und einem Wechselwirkungsterm $V_0$
welcher verschwindet wenn die Teilche weit voneinader entfernt seien.
\begin{equation}
		H = H_0 + V
\end{equation}
Wir verwenden das Heisenbergbild und definieren ein- $\ket{\alpha, +}$ und
auslaufende $\ket{\beta, -}$ Zustaende als Eigenzustaende des Hamiltonians,
\begin{equation}
		H \ket{\alpha, \pm} = E_{\alpha} \ket{\alpha, \pm}
\end{equation}
so, dass fuer Messungen bei $t \to \pm \infty$ wie Eigenzustaende des Hamiltonians
$H_0$ fure freie Teilchen $\ket{\alpha}_0$ erscheinen.
Die Zustaende $\ket{\alpha, +}$ und $\ket{\beta, -}$ bilden eine Basis desselben
Hilbertraums. Daher koennen wir $\ket{\alpha, +}$ und $\ket{\beta, -}$
ausdruecken als 
\begin{equation}
		\ket{\alpha, +} = \int d\beta \ S_{\beta \alpha} \ket{\beta, -}
\end{equation}
Damit folgt $S_{\beta \alpha}$ ist also die Uebergangsamplitude fuer den Prozess
$\ket{\alpha, +} \to \ket{\beta, -}$.
\begin{equation}
		\bra{\alpha, +} \ket{\beta, -} =S_{\beta \alpha}
\end{equation}
Da die ein- und auslaufende Zustaende orthogonal sind ist die S-Matrix
Orthogonal.
Aufgrund zeitlicher und raeumlicher Translationsinvarianz sind Gesamtenergie und
Impuls im Streuprozess erhalten, daher schreiben wir
\begin{equation}
		S_{\beta \alpha} = \delta_{\alpha \beta} - 2 \pi i \delta(E_{\alpha} -
		E_{\beta})\delta^3(\vec{p_{\alpha}}-\vec{p_{\beta}}) M_{\beta \alpha}
\end{equation}
Die Uebergangswahrscheinlichkeit $P$ ist dann proportional zu $T$, wir
bezeichnen den Koeffizienten als differentielle Uebergangsrate.
\begin{equation}
		d\Gamma(\alpha \to \beta) = \frac{dP(\alpha \to \beta)}{T}
\end{equation}
Ueblicherweise misst man die Uerbergangsrate normiert auf den Fluss
$\Phi_{\alpha} = u_{\alpha}/V$ der einlaufenden Teilchen.
Der differnzielle Wirkungsquerschnitt $d\sigma$ ist definiert als
\begin{equation}
		d\sigma(\alpha \to \beta) = \frac{d\Gamma(\alpha \to
		\beta)}{\Phi_{\alpha}}
\end{equation}
Der differentieller Wirkungsquerschnitt fuer $2 \to 2$ Streuung im CMS ist
gegeben durch
\begin{equation}
		\frac{d\sigma(\alpha \to \beta)}{d\Omega} = (2\pi)^4 \frac{E_1 E_2
		E_1' E_2'}{E^2} \frac{k'}{k} |M_{\alpha \beta}|^2
\end{equation}
\subsection{optisches Theorem}
\subsection{Stoerungstheorie}
Ziel ist die Berechnung der S-Matrix durch Potenzreihenentwicklung im Wechselwirkungsterm.
Dazu lassen sich die Zustaende schreiben als 
\begin{equation}
		\ket{\alpha, \pm} = \Omega(\mp \alpha)\ket{\alpha}_0
\end{equation}
mit 
\begin{equation}
		\Omega(t) = \exp(iHt) \cdot \exp(-iH_0t)
\end{equation}
Damit laesst sich die S-Matrix schreiben als 
\begin{eqnarray}
		\bra{\beta, -}\ket{\alpha, +} = \bra{\beta_0} \Omega^{\dag}(- \infty)
		\Omega^{\dag}(- \infty) \ket{\alpha_0}\\
		= \bra{\beta_0} U(+\infty, -\infty) \ket{\alpha_0}
\end{eqnarray}
wobei 
\begin{equation}
		\begin{split}
				U(t,t') = \exp(iH_0t) \exp(-iH(t-t')) \\ \cdot \exp(-iH_0t')
		\end{split}
\end{equation}
ist. Um U zu berechnen wird die Ableitung berechnet.
\begin{equation}
		\begin{split}
				\frac{d}{dt} U(t,t') = \exp(iH_0t) [H-H_0] \\ \cdot \exp(-iH(t-t')) \exp(-iH_0t')
		\end{split}
\end{equation}
wobei $H-H_0$ grade dem Potentail entspricht. Somit ergibt sich der Ausdruck
\begin{equation}
		\begin{split}
				\frac{d}{dt} U(t,t') = -i \exp(iH_0t) V \exp(-iH_0t) \\ 
				\cdot U(t, t') = -i V_I(t) U(t,t')
		\end{split}
\end{equation}
Aus der Bedingung das $U(t',t')=1$ sowie 
\begin{equation}
		U(t,t') = I_n -i\int^t_{t'} dt \ V_I(t)U(t,t')
\end{equation}
Durch iteratives einsetzen kann U aufgeloest werden. \begin{equation}
		\begin{split}
				U(t,t') = I_n -i\int^t_{t'} dt \ V_I(t) + (-i)^2 \\ \cdot \int^t_{t'}
				\int^{t_1}_{t'} dt_1 dt_2 \ V_I(t_1) V_I(t_2) + (-i)^3 \cdot (...)
		\end{split}
\end{equation}
Dies kann auch mittels des Zeitgeordneten Produkts geschrieben werden indem 
spaetere Zeiten links von frueheren Zeiten stehen.
\begin{eqnarray}
		T\{V(t)\} = V(t) \\
		\begin{split}
				T\{V(t_1)V(t_2)\} = \theta(t_1 -t_2)V(t_1)V(t_2) \\ + \theta(t_2 -t_1)V(t_2)V(t_1)
		\end{split}
\end{eqnarray}
Entsprechend laesst sich die S-Matrix zu 
\begin{equation}
		\begin{split}
				S_{\beta \alpha} = \sum^{\infty}_{n=0} \frac{(-i)^n}{n!} \int \ d^4x_1
				\dots d^4x_n \cdot \\ \bra{\beta_0} T\{H(x_1) \dots H(x_n)\} \ket{\alpha_0}
		\end{split}
\end{equation}
umschreiben.

\section{Quantenfelder}
Ziel ist Konstruktion einer Lorentzskalaren Wechselwirkungsdichte.
$\bra{p_1 \sigma_1 n_1, \ldots p_i \sigma_i n_i}$ bezeichnet den Zustand von
Nteilchen mit Impuls, Spin und Sorte. Die Teilchen sind endweder Fermionen oder
Boseonen so das sich das Vorzeichen rausziehen laesst.
Die Zustaende sind entsprechend
\begin{eqnarray}
		\text{N=0:} \braket{0}{0} =& 1 \\
		\text{N=1:} \braket{q'}{q} =& \delta_{p_1,p_2} \cdot
\delta_{\sigma_1,\sigma_2} \cdot \delta_{n_1,n_2}
\end{eqnarray}
definiert.
In aequivalenz zur HQM lassen sich die erzeuger und vernichter definieren.
\begin{eqnarray}
		a^{\dag}(q)\ket{q_1,q_2,q_3 \dots q_n} = \ket{q,q_1,q_2,q_3 \dots q_n}
		\\
		a^{\dag}(q) a^{\dag}(q_1) \dots a^{\dag}(q_n) \ket{0} = 
		\ket{q,q_1 \dots q_n}\\
		a(q) \bra{q} = \bra{0} \\
		a(q) \bra{-} = 0 
\end{eqnarray}
Entsprechend erfuellen die Operatoren $a^{dag}$ und $a$ die Vertauschungsreation
\begin{equation}
		\left[a(q),a^{\dag}(q')\right]= \delta(q-q')
\end{equation}
Alle Operatoren, die auf den Hilbertraum der Vielteilchenzustaende wirken lassen
sich als Summe von Produkten der Erzeuger und Vernichter schreiben. 
Insbesondere gilt das fuer ddie Wechseldichte H(x).
\subsection{Die Dirac-Algebra}
Die Erzeuger der Drehung $\vec{J}$ erfuellen die Vertauschungsrelation
$\left[J^i,J^j\right]=i\epsilon^{ijk}J^k$. 
Nimmt man den Lorentzboost $\vec{k}$ mit hinzu, erhaelt man zusaetzlich
\begin{eqnarray}
		\left[J^i,K^j\right]=i\epsilon^{ijk}K^k \\
		\left[K^i,K^j\right]=-i\epsilon^{ijk}J^k 
\end{eqnarray}
Mit $J=(J^{23}, J^{31}, J^{12})$ und $K=(J^{01}, J^{02}, J^{03})$ kann man dies
kompakt schreiben als 
\begin{equation}
		\begin{split}
		\left[ J^{\mu \nu}, J^{\rho \sigma} \right] = i \left( g^{\nu
		\rho}J^{\mu \sigma} - g^{\mu \rho}J^{\nu \sigma} \\ - g^{\sigma \mu} 
J^{\rho \nu} + g^{\sigma \nu}J^{\rho \mu} \right)
\end{split}
\end{equation}
wobei $J^{\mu \nu} = -J^{\nu \mu}$ die Erzeuger der Lorentzgruppe sind, d.h.
\begin{equation}
		D(\Lambda)=\exp\left(-i/2 \omega_{\mu \nu} J^{\mu \nu} \right)
\end{equation}
Ein Boost entlang einer Richtung entspricht somit einer Drehung um die
entsprechende Achse (\#Rapiditaet). 
Fuer die Konstruktion der Spin 1/2 Darstellung  werden die Gamma-Matrizen
definiert welche die Cliffordalgera erfuellen.
\end{document}
